%-----------------------------------LICENSE------------------------------------%
%   This file is free software: you can redistribute it and/or                 %
%   modify it under the terms of the GNU General Public License as             %
%   published by the Free Software Foundation, either version 3 of the         %
%   License, or (at your option) any later version.                            %
%                                                                              %
%   This file is distributed in the hope that it will be useful,               %
%   but WITHOUT ANY WARRANTY; without even the implied warranty of             %
%   MERCHANTABILITY or FITNESS FOR A PARTICULAR PURPOSE.  See the              %
%   GNU General Public License for more details.                               %
%                                                                              %
%   You should have received a copy of the GNU General Public License along    %
%   with this file.  If not, see <https://www.gnu.org/licenses/>.              %
%------------------------------------------------------------------------------%
%   Author: Ryan Maguire                                                       %
%   Date:   2023/12/31                                                         %
%------------------------------------------------------------------------------%
\documentclass{letter}
\signature{Ryan Maguire}
\address{%
    6188 Kemeny Hall\\%
    Mathematics Department\\%
    Dartmouth College\\%
    Hanover, NH USA 03755%
}
\begin{document}
    \begin{letter}{}
        \opening{To whom it may concern,}
            I am eager to apply for a research and teaching
            position at your university.
            I have deeply invested myself into mathematical pedagogy over the
            years and am ecstatic at the opportunity of teaching mathematics
            at the highest levels of education. My expertise in technical
            drawing and mathematical writing, and vast programming abilities
            and experience in computational and theoretical topology lead me to
            believe I would be a valuable addition to your department.
            \par
            In mathematics one learns by \textit{doing} and by \textit{seeing}.
            The best method of doing mathematics is to tackle problems and
            examples, guided by the instructor, and collaborating with peers.
            Seeing mathematics can be more of a challenge. To this end I've
            invested the past seven years to technical drawing in various
            programming languages (\texttt{asymptote},
            \texttt{tikz}, \texttt{POV-Ray}, \texttt{SVG}, and plain
            \texttt{C}) and have created over 700 figures to aid my students.
            This proves particularly useful in learning topology and analysis,
            where nearly every concept can be enhanced by the right figure.
            This has led to many \textit{ah-hah!} moments where the mathematics
            finally \textit{clicked} with my students, and I have been ever
            appreciative of my time spent with raytracing and rendering
            vector graphics.
            \par
            I am also very active in computational topology and programming
            with projects varying from Fourier analysis and signal processing
            to knot theory and abstract algebra. My primary work has been in
            knot tabulation and implementation of knot theoretic algorithms.
            Many such ideas stem from dynamical systems, for example using the
            Jones polynomial to differentiate between unstable periodic orbits
            (Gousbet \textit{et. al} studied this in 1999). These efforts have
            amassed a huge programming library called \texttt{libtmpl},
            released under the GNU GPLv3, that I have used to create very
            large knot invariant databases and process the radio science
            data from the NASA Cassini mission.
            \par
            Between UMass Lowell, Wellesley College, and Dartmouth College I
            have had the pleasure of being the teaching assistant for sixteen
            courses and the instructor for four, and have worked on a myriad of
            research projects in physics and mathematics. I would be more than
            excited to continue this work at your institution. If you have any
            questions, please do not hesitate to ask.
        \closing{Sincerely,}
    \end{letter}
\end{document}
