%-----------------------------------LICENSE------------------------------------%
%   This file is free software: you can redistribute it and/or                 %
%   modify it under the terms of the GNU General Public License as             %
%   published by the Free Software Foundation, either version 3 of the         %
%   License, or (at your option) any later version.                            %
%                                                                              %
%   This file is distributed in the hope that it will be useful,               %
%   but WITHOUT ANY WARRANTY; without even the implied warranty of             %
%   MERCHANTABILITY or FITNESS FOR A PARTICULAR PURPOSE.  See the              %
%   GNU General Public License for more details.                               %
%                                                                              %
%   You should have received a copy of the GNU General Public License along    %
%   with this file.  If not, see <https://www.gnu.org/licenses/>.              %
%------------------------------------------------------------------------------%
%   Author: Ryan Maguire                                                       %
%   Date:   2023/10/14                                                         %
%------------------------------------------------------------------------------%
\documentclass[a4paper,sans]{moderncv}
\usepackage[scale=0.75]{geometry}
\usepackage{moderncvbodyi, moderncvfooti}
\usepackage{multicol}
\usepackage{amssymb}

\moderncvstyle{casual}
\moderncvcolor{blue}
\name{Ryan}{Maguire}
\title{Curriculum Vitae}
\email{rmaguire@protonmail.com}
% \phone[mobile]{+1~(555)-555-5555}

\begin{document}
    \makecvtitle
    \vspace{-8ex}
    \section{Education}
        \cventry{2012--2016}
            {B.S.}
            {University of Massachusetts - Lowell}
            {}
            {}
            {Double Major in Mathematics and Physics}
        \cventry{2016--2017}
            {M.S.}
            {University of Massachusetts - Lowell}
            {}
            {}
            {Mathematics}
        \cventry{2019--2021}
            {M.A.}
            {Dartmouth College}
            {}
            {}
            {Mathematics}
        \cventry{2019--2024}
            {Ph.D.}
            {Dartmouth College}
            {}
            {}
            {Mathematics}
    \section{Experience}
        \cventry%
            {2024--Present}
                {Postdoctoral Associate}
                {MIT}
                {Cambridge, MA}
                {}
                {%
                    \begin{itemize}
                        \item
                            Instructor for the mathematics department.
                            \begin{itemize}
                                \item
                                    18.100P: Real Analysis (Spring 2025).
                            \end{itemize}
                        \item
                            Designing a vector calculus course for the
                            open learning platform.
                        \item
                            Researching topology, knot theory,
                            and computational mathematics.
                        \item
                            Directing two PRIMES groups (Jan-Dec 2025). The
                            subjects are:
                            \begin{itemize}
                                \item
                                    Computational knot theory.
                                \item
                                    Applied analysis and astronomy
                                    (in collaboration with Dick French
                                    from Wellesley College).
                            \end{itemize}
                        \item
                            Supervising research for undergraduates
                            (MIT UROP) in applied analysis,
                            mathematical physics, and astronomy.
                    \end{itemize}
                }
        \cventry%
            {2021--2023}
            {Lecturer}
            {Dartmouth College}
            {Hanover, NH}
            {}
            {%
                \begin{itemize}
                    \item Math 3, calculus (Fall 2021)
                    \item Math 54, point-set topology (Summer 2022, 2023)
                    \item Math 87, Topics in Topology (Fall 2022)
                \end{itemize}
            }
        \cventry%
            {2020--2024}
            {Research Assistant}
            {Dartmouth College}
            {Hanover, NH}
            {}
            {%
                Worked in topology,
                particular knot theory and Lorentz geometry.
                \begin{itemize}
                    \item
                        Designed and implemented algorithms for the
                        computations of knot invariants.
                    \item
                        Created large databases of knot invariants for studying
                        and experimenting with conjectures.
                \end{itemize}%
             }
        \cventry%
            {2019--2021}
            {Teaching Assistant}
            {Dartmouth College}
            {Hanover, NH}
            {}
            {%
                \begin{itemize}
                    \item
                        Math 3 (Differential Calculus).
                    \item
                        Math 8 (Differential and Integral Calculus).
                    \item
                        Math 13 (Vector Calculus).
                    \item
                        Math 17 (Knot Theory and Geometry).
                    \item
                        Math 32 (The Shape of Space).
                \end{itemize}
            }
        \cventry%
            {2017--Present}
            {Research Assistant}
            {Wellesley College}
            {Wellesley, MA}
            {}
            {%
                Working with the Cassini Radio Science
                Experiment studying ring occultations.
                \begin{itemize}
                    \item
                        Work in diffraction theory studying Fredholm
                        equations, Fresnel inversion, Fourier optics,
                        and signal processing.
                     \item
                        Develop open source software in Python and C to
                        study the radio science data from the
                        Cassini mission to Saturn.
                 \end{itemize}
                 Studied surgery theory and topology.
                 \begin{itemize}
                     \item
                        Participated in an informal reading
                        course in algebraic topology.
                     \item
                        Attended a series of lectures on Surgery Theory.
                 \end{itemize}%
             }
        \cventry%
            {2012--2017}
            {Research Assistant}
            {Lowell Center for Science and Space Technology}
            {Lowell, MA}
            {}
            {%
                Worked in spectroscopy, interferometry,
                LIDAR, and data analysis.
                \begin{itemize}
                    \item
                        Worked on DWEL, a dual wavelength LIDAR that was
                        deployed in Australia, California, and Harvard Forest.
                    \item
                        Constructed and deployed HiT\&MIS.
                        Instrument was successfully deployed to Kiruna,
                        Sweden working with Dartmouth College's BARREL
                        Mission. Performed data analysis on the airglow of
                        several events.
                    \item
                        Worked on the data from the SPINR sounding rocket.
                        Developed code in IDL and worked on archiving the
                        data for MAST.
                \end{itemize}%
             }
        \cventry%
            {2015--2019}
            {Teaching Assistant}
            {University of Massachusetts - Lowell}
            {Lowell, MA}
            {}
            {%
                \begin{itemize}
                    \item
                        Management Pre-Calculus.
                    \item
                        Differential Calculus.
                    \item
                        Differential Equations.
                    \item
                        Electromagnetism.
                    \item
                        Mathematics tutor for disability services.
                    \item
                        General tutoring for the physics department.
                    \item
                        General tutoring for the mathematics department.
                \end{itemize}%
            }
    \section{Interests}
        \makebox[2.3cm][l]{Analysis}
            Fourier and functional analysis, numerical methods.\par
        \makebox[2.3cm][l]{Geometry}
            Contact geometry, Lorentz geometry, spacetimes.\par
        \makebox[2.3cm][l]{Physics}
            Cosmology, general relativity, particle physics.\par
        \makebox[2.3cm][l]{Topology}
            Contact topology, knot theory, point-set.
    \section{Publications}
        \begin{enumerate}
            \item
                \textit{Derivation of the Energy and Flux Morphology in an
                Aurora Observed at Midlatitude Using Multispectral Imaging},
                JGR Space Physics, 2018, with Saurav Aryal \textit{et al.}
            \item
                \textit{The Seven-lobed Shape of the Outer
                Edge of Saturn’s A Ring}, Icarus, volume 390, 2023, with
                Philip D. Nicholson, \textit{et al.}
            \item
                \textit{The complex shape of the outer edge of Saturn’s B ring,
                as observed in Cassini occultation data},
                Icarus, volume 405, 2023, with Richard French \textit{et al.}
            \item
                \textit{%
                    Affine linking number estimates for the number of
                    times an observer sees a star%
                },
                Class. Quantum Grav., Vol. 41, 2024, with Vladimir Chernov.
            \item
                \textit{%
                    Further Reverberations of the 1983
                    Impact with Saturn’s C Ring
                },
                2025, with Richard French \textit{et al.}
        \end{enumerate}
    \section{Pre-Prints}
        \begin{enumerate}
            \item
                \textit{%
                    Properties of an infinite dimensional
                    Banach space over the field with two elements.%
                },
                \textit{arxiv}, 2019, with Samuel Gomez and James Rose.
            \item
                \textit{%
                    Conjectures on the Khovanov Homology of
                    Legendrian and Transversely Simple Knots%
                },
                \textit{arxiv}, 2023, with Vladimir Chernov.
        \end{enumerate}
    \section{Presentations}
        \begin{enumerate}
            \item UML Symposium (2014)
            \item AGU Fall Meeting (2014):
            \item CEDAR Workshop (2015)
            \item AGU Fall Meeting (2015)
            \item CEDAR Workshop (2017)
            \item NEROC Symposium (2018)
            \item Final Cassini Symposium (2018)
            \item
                Dartmouth Graduate Student Seminar (2019):
                Diaconescu's Theorem.
            \item
                Dartmouth Graduate Student Seminar (2019):
                Stone's Representation Theorem.
            \item
                Dartmouth Graduate Student Seminar (2020):
                Semi-Riemannian Geometry.
            \item
                Dartmouth Topology Seminar (Spring 2022):
                The Jones Polynomial
            \item
                Knots in Washington (Spring 2022):
                An Algorithm for the Jones Polynomial.
            \item
                Dartmouth Informal Seminar on Topology and Geometry (2022):
                Affine Connections.
            \item
                Dartmouth Informal Seminar on Topology and Geometry (2022):
                Semi-Riemannian Manifolds.
            \item
                Dartmouth Informal Seminar on Topology and Geometry (2022):
                Homotopy Groups of Spheres.
            \item
                Dartmouth Informal Seminar on Topology and Geometry (2022):
                Einstein Equations (Part 1).
            \item
                Dartmouth Informal Seminar on Topology and Geometry (2022):
                Einstein Equations (Part 2).
            \item
                Dartmouth Graduate Student Seminar (2023):
                Pade Approximates and the Remez Algorithm.
            \item
                Dartmouth Graduate Student Seminar (2023):
                Newtonian Black Holes.
            \item
                Dartmouth Informal Seminar on Topology and Geometry (2023):
                The Low Conjecture.
            \item
                Dartmouth Informal Seminar on Topology and Geometry (2023):
                Borde-Sorkin Spacetimes.
            \item
                Knots in Washington (2023):
                Conjectures on Legendrian Simple Knots.
            \item
                Dartmouth Informal Seminar on Topology and Geometry (2023):
                Black Holes.
            \item
                Dartmouth Graduate Student Seminar (2023):
                Legendre Polynomials and Saturn.
            \item
                Knots in Washington (2023):
                Gravitational Lensing.
            \item
                Thesis Defense (2024):
                Khovanov Homology and Legendrian Simple Knots
            \item
                USF Topology Seminar (2024):
                Relative Strengths of Knot Invariants by Experiment
            \item
                CMC Topology Seminar (2024):
                Relative Strengths of Knot Invariants by Experiment
            \item
                Trisectors Workshop (2024):
                Exotic Smooth Structures and Knot Traces.
            \item
                AMS Fall Northeast Sectionals (2024):
                Tait Graphs of Virtual Knots.
            \item
                Knots in Washington 50 (2024):
                Tait Graphs of Virtual Knots.
        \end{enumerate}
    \section{Coding Projects}
        \begin{itemize}
            \item
                \texttt{libtmpl}
                (\url{https://github.com/ryanmaguire/libtmpl/})
                \begin{itemize}
                    \item
                        Nearly complete implementation of \texttt{libm}, the
                        C standard mathematical library, with additional
                        features such as vector geometry, algebra, knot theory,
                        Fourier transforms, complex numbers, and more.
                \end{itemize}
            \item
                \texttt{rss\_ringoccs}
                (\url{https://github.com/NASA-Planetary-Science/rss_ringoccs/})
                \begin{itemize}
                    \item
                        Suite of code for processing and analyzing the radio
                        science data from the NASA Cassini mission. This
                        numerically inverts the data from occultation
                        observations and can be used to study the rings of
                        Saturn. Recent improvements allow the code to be used on
                        Uranus as well.
                \end{itemize}
            \item
                \texttt{Mathematics-and-Physics}
                (\url{https://github.com/ryanmaguire/Mathematics-and-Physics/})
                \begin{itemize}
                    \item
                        All of my course notes and code for figures. This
                        includes 520+ drawings and animations in asymptote,
                        130+ in tikz, and a handful in other languages such as
                        POV-Ray and C.
                \end{itemize}
            \item
                \texttt{barnsley\_fern}
                (\url{https://github.com/ryanmaguire/barnsley_fern/})
                \begin{itemize}
                    \item
                        Code for rendering the Barnsley fern and variants.
                    \item
                        Implementations: \texttt{C, C++, Python, Go}
                \end{itemize}
            \item
                \texttt{complex\_visual\_plots}
                (\url{https://github.com/ryanmaguire/complex_visual_plots/})
                \begin{itemize}
                    \item
                        Tools for drawing plots of complex functions. Capable
                        of visualizing Newton and Mandelbrot iterations for
                        complex functions.
                    \item
                        Implementations: \texttt{C, C++, Go}
                \end{itemize}
            \item
                \texttt{mandelbrot\_set}
                (\url{https://github.com/ryanmaguire/mandelbrot_set/})
                \begin{itemize}
                    \item
                        Renderings of the Mandelbrot set and its variants.
                    \item
                        Implementations: \texttt{C, Python}
                \end{itemize}
            \item
                \texttt{metric\_space\_disks}
                (\url{https://github.com/ryanmaguire/metric_space_disks/})
                \begin{itemize}
                    \item
                        Draws the closed unit ball for arbitrary metrics in the
                        plane. The $L_{1}$, $L_{2}$, and $L_{\infty}$ metrics
                        are provided as examples.
                    \item
                        Implementations: \texttt{C}
                \end{itemize}
            \item
                \texttt{newton\_fractals}
                (\url{https://github.com/ryanmaguire/newton_fractals/})
                \begin{itemize}
                    \item
                        Renders Newton fractals for arbitrary polynomials
                        $p\in\mathbb{C}[z]$. Contains a full implementation of
                        the Hubbard-Schleicher-Sutherland algorithm for
                        determining all of the roots of complex polynomials.
                    \item
                        Implementations: \texttt{C, Python}
                \end{itemize}
            \item
                \texttt{newtonian\_black\_holes}
                (\url{https://github.com/ryanmaguire/newtonian_black_holes/})
                \begin{itemize}
                    \item
                        Raytraces black holes by abusing Newtonian mechanics.
                        The force of gravity is allowed to act on light as if
                        photons had an arbitrarily small non-zero mass.
                        Numerical methods are used to raytrace the path of
                        light.
                    \item
                        Implementations:
                        \texttt{%
                            C, C++, Fortran, Go, IDL, Java,
                            Pascal, Python, Rust, Swift%
                        }
                \end{itemize}
            \item
                \texttt{perfectly\_normal\_spaces}
                (\url{https://github.com/ryanmaguire/perfectly_normal_spaces/})
                \begin{itemize}
                    \item
                        Visual for the fact that $\mathbb{R}^{2}$ is a perfectly
                        normal topological space. Given any two disjoint closed
                        regions $\mathcal{C}$, $\mathcal{D}$, that are each the
                        finite union of polygonal figures, this plots the
                        separating function:
                        \begin{equation}
                            f\big((x,\,y)\big)
                            =\frac{\textrm{dist}\big((x,\,y),\,\mathcal{C}\big)}
                            {\textrm{dist}\big((x,\,y),\,\mathcal{C}\big)+
                             \textrm{dist}\big((x,\,y),\,\mathcal{D}\big)}
                        \end{equation}
                        The example provided are the Chinese symbols
                        \textit{tu} and \textit{dou}, which translates to
                        \textit{potato} in English, a running joke in my
                        topology class during the summer of 2022.
                    \item
                        Implementations: \texttt{C}
                \end{itemize}
            \item
                \texttt{random\_walks}
                (\url{https://github.com/ryanmaguire/random_walks/})
                \begin{itemize}
                    \item
                        Draws random walks on various manifolds. Current
                        manifolds provided are the torus, Klein bottle, and
                        real projective plane.
                    \item Implementations: \texttt{C, asymptote}
                \end{itemize}
        \end{itemize}
    \section{Professional Development}
        \begin{itemize}
            \begin{multicols}{2}
                \item ITAR Control Module, January 2013
                \item Laboratory Safety Training, January 2013
                \item DCAL Teaching Training, Winter 2019-2020
            \end{multicols}
        \end{itemize}
    \section{Affiliations/Memberships}
        \begin{itemize}
            \begin{multicols}{2}
                \item Society of Physics Students
                \item Graduate Physics Association
                \item American Mathematical Society
            \end{multicols}
        \end{itemize}
    \section{Languages}
        \makebox[2.3cm][l]{Native} English\par
        \makebox[2.3cm][l]{Elementary} French, German, Turkish
    \section{Computer skills}
        \makebox[2.3cm][l]{Advanced}
        \texttt{C, Python, Linux/Unix, Bash}, \LaTeX\par
        \makebox[2.3cm][l]{Intermediate}
        \texttt{C++, Fortran, Rust, Pascal, Go, IDL, Swift, Java, SageMath}\par
        \makebox[2.3cm][l]{Elementary}
        \texttt{Mathematica, MATLAB, Ruby, Julia}
    \section{Technical Drawing}
        \makebox[2.3cm][l]{Advanced}
        \texttt{asymptote}\par
        \makebox[2.3cm][l]{Intermediate}
        \texttt{tikz}\par
        \makebox[2.3cm][l]{Elementary}
        \texttt{POV-Ray, SVG}
\end{document}
