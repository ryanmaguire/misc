%-----------------------------------LICENSE------------------------------------%
%   This file is free software: you can redistribute it and/or                 %
%   modify it under the terms of the GNU General Public License as             %
%   published by the Free Software Foundation, either version 3 of the         %
%   License, or (at your option) any later version.                            %
%                                                                              %
%   This file is distributed in the hope that it will be useful,               %
%   but WITHOUT ANY WARRANTY; without even the implied warranty of             %
%   MERCHANTABILITY or FITNESS FOR A PARTICULAR PURPOSE.  See the              %
%   GNU General Public License for more details.                               %
%                                                                              %
%   You should have received a copy of the GNU General Public License along    %
%   with this file.  If not, see <https://www.gnu.org/licenses/>.              %
%------------------------------------------------------------------------------%
%   Author: Ryan Maguire                                                       %
%   Date:   2023/12/31                                                         %
%------------------------------------------------------------------------------%
\documentclass{article}
\title{Diversity Statement}
\author{Ryan Maguire}
\date{\today}
\begin{document}
    \maketitle
    I truly believe that anyone can pursue the study of mathematics. The
    difficulty of this pursuit is not constant, however, and varies greatly
    from person to person. Among these difficulties are the biases and
    discrimination imposed upon one by their origin. The impediment that
    has most impacted me is that of poverty and I have made great efforts to
    try and combat this. I do not believe that financial status should bar ones
    access to knowledge, especially a study as beautiful as mathematics.
    \par\hfill\par
    Not being a person of substantial wealth myself, my effort to help the
    less fortunate in mathematics has been aimed at access to materials.
    It is a complete fiction to believe that a \$200 calculus textbook is
    \textit{required} to better understand the ideas. The source materials for
    the subject were published between the
    $17^{\textrm{\tiny{th}}}$ and $19^{\textrm{\tiny{th}}}$ centuries and
    are all long within the public domain. Still, they do not read easily to
    the modern student, the language can be fuzzy at times, but I do not believe
    the cure is an overpriced textbook. To this end I have endeavored
    to write my own notes, my own problem sets, and my own examples,
    whenever I teach a course. This has been an extremely laborious task,
    but if it results in a 50+ person class saving hundreds of dollars each,
    then I am satisfied.
    \par\hfill\par
    My efforts in this pursuit have not solely resided in the introductory
    courses where the cost of textbooks are most egregious. The cost of a
    topology textbook was \$70 when I first taught it. I objected to requiring
    the book and during a discussion over this I recall one of the other
    instructors proclaiming \textit{oh, it's just 70 dollars}. A hint of anger
    entered the back of my mind, 70 dollars can be more than a week of food for
    the poorer student. I did not relent and I am quite pleased that the
    textbook was not required for the class. I am particularly proud of the
    efforts I undertook during this semester, as it culminated in over 200 pages
    of material made freely available for all.
\end{document}
