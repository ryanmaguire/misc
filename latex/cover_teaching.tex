%-----------------------------------LICENSE------------------------------------%
%   This file is free software: you can redistribute it and/or                 %
%   modify it under the terms of the GNU General Public License as             %
%   published by the Free Software Foundation, either version 3 of the         %
%   License, or (at your option) any later version.                            %
%                                                                              %
%   This file is distributed in the hope that it will be useful,               %
%   but WITHOUT ANY WARRANTY; without even the implied warranty of             %
%   MERCHANTABILITY or FITNESS FOR A PARTICULAR PURPOSE.  See the              %
%   GNU General Public License for more details.                               %
%                                                                              %
%   You should have received a copy of the GNU General Public License along    %
%   with this file.  If not, see <https://www.gnu.org/licenses/>.              %
%------------------------------------------------------------------------------%
%   Author: Ryan Maguire                                                       %
%   Date:   2023/12/31                                                         %
%------------------------------------------------------------------------------%
\documentclass{letter}
\signature{Ryan Maguire}
\address{%
    6188 Kemeny Hall\\%
    Mathematics Department\\%
    Dartmouth College\\%
    Hanover, NH USA 03755%
}
\begin{document}
    \begin{letter}{}
        \opening{To whom it may concern,}
            I am eager to apply for a teaching position at your university.
            I have deeply invested myself into mathematical pedagogy over the
            years and am ecstatic at the opportunity of teaching mathematics
            at the highest levels of education. My expertise in technical
            drawing and mathematical writing leads me to believe I can be a
            valuable addition to your department.
            \par
            In mathematics one learns by \textit{doing} and by \textit{seeing}.
            The best method of doing mathematics is to tackle problems and
            examples, guided by the instructor, and collaborating with peers.
            Seeing mathematics can be more of a challenge. To this end I've
            invested the past seven years to technical drawing in various
            programming languages (\texttt{asymptote},
            \texttt{tikz}, \texttt{POV-Ray}, \texttt{SVG}, and plain
            \texttt{C}) and have created over 700 figures to aid my students.
            This proves particularly useful in learning topology and analysis,
            where nearly every concept can be enhanced by the right figure.
            This has led to many \textit{ah-hah!} moments where the mathematics
            finally \textit{clicked} with my students, and I have been ever
            appreciative of my time spent with raytracing and rendering
            vector graphics.
            \par
            Aside from contributing to mathematical visualization, I believe
            access to knowledge should not be barred by financial constraints.
            In my topology course I wrote well over 200 pages of notes for the
            students with detailed proofs and examples (and containing many
            figures, of course). I have made similar efforts in calculus and
            electromagnetism courses, receiving the thanks of both students
            and other instructors. I have made the source code for all of my
            figures and writings available as free software, released under the
            GPL3, in the hopes of combating the monetary constrains
            often imposed on students.
            \par
            Between UMass Lowell and Dartmouth College I have had the pleasure
            of being a TA for sixteen courses and the instructor of four. In
            this time I've met several amazing students and feel I've left a
            positive influence on them. I would be grateful
            for the opportunity to further this work at your institution.
            Should you have any questions, please do not hesitate to ask.
        \closing{Sincerely,}
    \end{letter}
\end{document}
